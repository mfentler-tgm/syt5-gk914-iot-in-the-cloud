\section{Verwendete Technologien}

Es werden folgende Technologien verwendet:

\begin{itemize}
    \item Raspberry Pi Zero W: Weil er klein und kostengünstig ist
    \item Node.JS: für den Webserver (Livestream), GPIO Zugriff (Bewegungssensor), Bilder, Mail
\end{itemize}

Verwendete Node.JS libraries:

\begin{itemize}
    \item express: Für den Webserver/Livestream
    \item hls.js: Livestream in HLS format
    \item ip: IP-Addresse auslesen
    \item nodemailer: Mail mit IP/Bild senden
    \item onoff: Für GPIO pins
    \item pi-camera: Um auf die Kamera zuzugreifen
\end{itemize}

Um den Schaltplan zu verfassen wird die Desktop-Applikation \textbf{Fritzing} verwendet.

\clearpage
\section{Research}

\subsection{Raspberry Setup}

Die initiale Konfiguration des Raspberry Pi's erfolgt mittels der \texttt{raspi-info.md} Dokumentation.

\subsection{Abtastrate der Informationen}

Da eine fixe Abtastrate ineffizient wäre, wird ein GPIO callback verwendet um ein Bild zu senden.

Der Bewegungssensor wird folgendermaßen initialisiert:

\begin{listing}
    \begin{code}{js}
    const pir = new Gpio(17, 'in', 'both')
    \end{code}{js}
    \caption{GPIO Erstellung in JavaScript}
\end{listing}

Mittels dem erstellten \texttt{Gpio} Objekt kann ein Callback registriert werden:

\begin{listing}
    \begin{code}[firstnumber=last]{js}
    pir.watch(function(err, value) {
        // Bewegungsmelder callback
        if (err) exit()
        console.log('Intruder detected')
        if (value == 1) {
            mail.sendMail()
        }
    })
    \end{code}{js}
    \caption{GPIO Callback registrierung mittels watch in JavaScript}
\end{listing}

Zusätzlich wird ein Livestream über die Website erstellt, wobei keine fixe Abtastrate verwendet wird.

\clearpage
\subsection{Aggregierung der Daten}
Bei jeder Rising Edge des Bewegungssensors wird eine Mail an den Besitzer gesendet. Der Code schaut folgendermaßen aus:
\begin{listing}
    \begin{code}[firstnumber=last]{js}
    const pir = new Gpio(17, 'in', 'both')
    pir.watch(function(err, value) {
        if (err) exit()

        console.log('Intruder detected')

        if (value == 1) {
            mail.sendMail()
        }
    })
    \end{code}{js}
    \caption{Sender der Mail bei Erkennung einer Bewegung bzw. Rising Edge des Sensors}
\end{listing}

\subsection{Schnittstellendefinition}

\subsection{Energieversorgung}

Der Raspberry Pi benötigt eine Stromversorgung über Micro-USB, welche entweder über eine normale Stromanbindung über eine Steckdose oder über ein Akku-pack erreicht wird.

\subsection{Speicherverbrauch}

Das \textbf{Basement Shitting Prevention} Programm benötigt \texttt{2.552} Bytes. Zusätzlich werden einige Packages benötigt, womit ein insgesamter Speicherverbrauch von \texttt{idk} Megabytes verwendet wird.

Der Raspberry Pi ist mit einer \texttt{32} GB SD-Karte ausgestattet, wobei neben Betriebssystem und Programm genügend Speicherplatz vorhanden bleibt.

\subsection{Verbindung}




\clearpage
\section{Implementierung}
