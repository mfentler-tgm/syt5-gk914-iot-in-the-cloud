\section{Verwendete Technologien}

Es werden folgende Technologien verwendet:

\begin{itemize}
    \item Raspberry Pi Zero W: Weil er klein und kostengünstig ist
    \item Node.JS: für den Webserver (Livestream), GPIO Zugriff (Bewegungssensor), Bilder, Mail
\end{itemize}

Verwendete Node.JS libraries:

\begin{itemize}
    \item express: Für den Webserver/Livestream
    \item hls.js: Livestream in HLS format
    \item ip: IP-Addresse auslesen
    \item nodemailer: Mail mit IP/Bild senden
    \item onoff: Für GPIO pins
    \item pi-camera: Um auf die Kamera zuzugreifen
\end{itemize}

Um den Schaltplan zu verfassen wird die Desktop-Applikation \textbf{Fritzing} verwendet.

\clearpage
\section{Research}

\subsection{Raspberry Setup}

Die initiale Konfiguration des Raspberry Pi's erfolgt mittels der \texttt{raspi-info.md} Dokumentation.

\subsection{Abtastrate der Informationen}

\subsection{Aggregierung der Daten}

\subsection{Schnittstellendefinition}

\subsection{Energieversorgung}

\subsection{Speicherverbrauch}

\subsection{Verbindung}



\clearpage
\section{Implementierung}
